\documentclass[11pt,a4paper,sans]{moderncv}

\moderncvstyle{classic}
\moderncvcolor{blue}

\usepackage[scale=0.8]{geometry}

\ifxetexorluatex
  \usepackage{fontspec}
  \usepackage{unicode-math}
  \defaultfontfeatures{Ligatures=TeX}
  \setsansfont{Latin Modern Sans}
  \setmonofont{Latin Modern Mono}
  \setmathfont{Latin Modern Math} 
\else
  \usepackage[utf8]{inputenc}
  \usepackage[T1]{fontenc}
  \usepackage{lmodern}
\fi

% document language
\usepackage[english]{babel}

% make all links blue
\usepackage[unicode]{hyperref}
\definecolor{linkcolour}{rgb}{0,0.2,0.6}
\hypersetup{colorlinks=true,urlcolor=linkcolour,linkcolor=linkcolour}

\name{Denis}{Shkiria}
\address{}{Belgrade}{Serbia}
\email{denis.shkirja@gmail.com}
\born{24 March 1991}

\begin{document}
\makecvtitle

\section{Experience}
\cventry
{March 2021 -- August 2023}
{C++ Developer, Backend Team}
{\httpslink[ZION Development]{devzion.com}}
{Saint-Petersburg/Tbilisi}
{}
{Development of C++ microservices for a financial system that facilitates the exchange of cryptocurrencies.\newline{}
Achievements:
\begin{itemize}
\item Implementation of a protocol for requesting data between C++ services. The requests are then converted to data queries using Boost.HANA and Boost.mp11.
\item Covering existing code with unit tests, adding new service contracts and documenting them. Improving the reliability and performance of an internal framework used for developing microservices.
\item Solving production problems using logs and metrics.
\item Development of a wrapper library over \httpslink[V8]{v8.dev} JS Engine for the ability to implement part of the business logic of C++ microservices in Java script.
  \begin{itemize}
  \item Support of the 'import' keyword in JS code and implementation of module loading on the C++ side.
  \item Implementation of limits on CPU and wall execution time of JS code for JS handlers.
  \item Implementation of a convenient mechanism for passing functions and types from C++ to JS for use as an API, including for asynchronous functions.
  \end{itemize}
\item Development of an internal library for connecting to exchanges and adding support for new exchanges (for example, \httpslink[OKX]{okx.com}) to it.
\end{itemize}}

\cventry
{March 2018 -- February 2021}
{C++ Developer, Connectivity team}
{\httpslink[Itiviti]{www.broadridge.com/financial-services/capital-markets/trading-and-connectivity/}(now Broadridge)}
{Saint-Petersburg}
{}
{Support and development of the components responsible for connecting to the markets. Most often I worked with the \httpslink[T7]{eurexchange.com/exchange-en/technology/t7} platform and its protocols.\newline{}
Achievements:
\begin{itemize}
\item Support of updates and new features (for example, \httpslink[Eurex Enlight]{eurex.com/ex-en/trade/enlight}) in T7 protocols, bug fixes and refactoring of existing code.
\item Assist account managers in solving customer problems.
\item Development of internal utilities for archiving and restoring audit data generated by system components.
\end{itemize}}

\cventry
{July 2015 -- March 2018}
{C++ Developer, Core team}
{\httpslink[Itiviti]{www.broadridge.com/financial-services/capital-markets/trading-and-connectivity/}(now Broadridge)}
{Saint-Petersburg}
{}
{Support and development of internal libraries. Test writing, refactoring, core dump analysis.\newline{}
Achievements:
\begin{itemize}
\item Support for \httpslink[Join]{source.wiredtiger.com/11.1.0/cursor\_join.html} cursors in an internal storage library, which is a wrapper over the Wiredtiger and other storage engines.
  \begin{itemize}
  \item Publish bug reports (\httpslink[1]{jira.mongodb.org/browse/WT-2308}, \httpslink[2]{jira.mongodb.org/browse/WT-2447}, \httpslink[3]{jira.mongodb.org/browse/WT-2414}, ...) with reproduction tests for the Wiredtiger team.
  \end{itemize}
\item Improvement of the functionality related to forced termination of user strategies in case of their freezing.
  \begin{itemize}
  \item Step-by-step unwinding of a stack of an interrupted thread using the \httpslink[libunwind]{www.nongnu.org/libunwind/} library to exit strategy code and resume execution from a safe point in an engine code.
  \end{itemize}
\item Work in a project to meet the requirements of the \httpslink[MIFID2]{en.wikipedia.org/wiki/Markets\_in\_Financial\_Instruments\_Directive\_2004\#MiFID\_II.2FMiFIR} regulation.
  \begin{itemize}
  \item Implementation of intermediate audit data storage required by the regulator.
  \item Implementation of price and volume pre-trade limits for quotas.
  \end{itemize}
\end{itemize}}

\cventry
{June 2010 -- July 2015}
{C Developer}
{\httpslink[Neobit]{neobit.ru}}
{Saint-Petersburg}
{}
{I worked here part-time until June 2012 and participated in various projects, mainly using C, sometimes - C++, asm x86, bash, C\#.\newline{}
Achievements:
\begin{itemize}
\item Development of a thin type 2 hypervisor based on VT-x hardware virtualization technology.
  \begin{itemize}
  \item Correct loading of guest OS under control of the hypervisor.
  \item Nested hardware virtualization for VT-x.
  \item Interaction with the guest OS from the hypervisor (Windows).
  \item Emulation of some x86 instructions.
  \end{itemize}
\item Participation in the development of the microkernel OS ’Febos’.
  \begin{itemize}
  \item The initial stage of loading.
  \item Search and fix errors in the hypervisor and device emulators used for running a guest OS.
  \end{itemize}
\item Preparation of tasks for the annual hackathon organized by the company - \httpslink[NeoQUEST]{neoquest.ru}. Task solutions were later posted in the company’s blog. In addition, I wrote articles not related to NeoQUEST (\httpslink[1]{habr.com/ru/company/neobit/blog/162769},
								\httpslink[2]{habr.com/ru/company/neobit/blog/173263},
								\httpslink[3]{habr.com/ru/company/neobit/blog/211470},
								\httpslink[4]{habr.com/ru/company/neobit/blog/203706},
								\httpslink[5]{habr.com/ru/company/neobit/blog/197912},
								\httpslink[6]{habr.com/ru/company/neobit/blog/261081}).
\item Development of a prototype of a \httpslink[CSP]{en.wikipedia.org/wiki/Cryptographic\_Service\_Provider} cryptographic provider.
\item Modification of the VirtualBox code to interact with the guest OS driver from the host side.
\end{itemize}}

\cventry
{September 2008 -- February 2011}
{Manual tester}
{LGPolyREC}
{Saint-Petersburg}
{}
{I manually tested applications developed at the local LG office using pre-built and custom test cases.}

\section{Education}
\cventry
{2007 -- 2012}
{Information security of automated systems}
{\newline{} Peter the Great Polytechnic University, Institute of Cybersecurity and Information Protection}
{Saint Petersburg}
{}
{}

\section{Languages}
\cvitemwithcomment{Russian}{Native}{}
\cvitemwithcomment{English}{B2}{}

\end{document}
