\documentclass[11pt,a4paper,sans]{moderncv}

\usepackage[T2A]{fontenc}
\usepackage[utf8]{inputenc}
\usepackage[scale=0.8]{geometry}
\moderncvstyle{classic}
\moderncvcolor{purple}

\usepackage[unicode]{hyperref}
\definecolor{linkcolour}{rgb}{0,0.2,0.6}
\hypersetup{colorlinks,breaklinks,urlcolor=linkcolour, linkcolor=linkcolour}

\name{Шкиря}{Денис}
\address{}{Санкт-Петербург, Россия}
\phone[mobile]{8 911 987 0340}
\email{denis.shkirja@gmail.com}
\social[github]{denisShkirja}
\extrainfo{Дата рождения: 24 марта 1991}

\begin{document}
\makecvtitle

\section{Цель}
\cvitem
    {}
    {Работа программистом С/С++ в продуктовой компании с налаженным процессом
    разработки.}

\section{Образование}
    \cventry
        {Сентябрь 2007 -- Сентябрь 2012}
        {Комплексное обеспечение информационной безопасности автоматизированных систем}
        {\newline Санкт-Петербургский государственный политехнический университет,
        Факультет технической кибернетики, кафедра ИБКС}
        {}
        {Специалист}
        {}
    \subsection{Дипломная работа}
        \cvitem{Название}{\emph{Интеграция криптопровайдера в продукты MS Office}}
        \cvitem{Руководитель}{Зегжда Дмитрий Петрович}
        \cvitem{Описание}{Целью работы была разработка криптопровайдера (CSP), на базе
            которого можно было бы сгенерировать самоподписанный сертификат
            стандартными утилитами Windows. Кроме того, нужно было научиться добавлять в
            документы MS Office электронную подпись, сгенерированную на основе полученного
            сертификата.}

\section{Опыт работы}
    \cventry
        {Июнь 2010 -- Настоящее время}
        {Программист С/C++}
        {Необит}
        {Санкт-Петербург}
        {}
        {До июня 2012 года работал на неполной ставке.
        \newline Основной язык - C, дополнительные - C++, asm x86, bash, C\#.\newline{}%
            Задачи:%
            \begin{itemize}%
            \item Разработка тонкого гипервизора на основе технологий
                аппаратной виртуализации (VT-x, SVM).
                \begin{itemize}
                \item Корректная загрузка гостевых ОС парарельно с работой
                    гипервизора.
                \item Вложенная аппаратная виртуализация VT-x.
                \item Взаимодействие с гостевой ОС (Windows).
                \item Эмуляция инструкций.
                \end{itemize}
            \item Разработка микроядерной ОС 'Фебос'.
                \begin{itemize}%
                \item Начальный этап загрузки.
                \item Поиск и исправление ошибок в гипервизоре и эмуляторах
                    устройств при запуске гостевых ОС.
                \item Профайлинг ядра.
                \end{itemize}
            \item Подготовка заданий для очных и заочных туров NeoQUEST.
                \begin{itemize}%
                \item К заданиям писались прохождения и выкладывались в виде
                    статей в блоге компании на Habrahabr.ru.
                \end{itemize}
            \item Разработка прототипа криптопровайдера (CSP).
            \item Модификация кода VirtualBox для взаимодействия с драйвером
                гостевой ОС со стороны хоста.
            \end{itemize}
        }
    \cventry
        {Сентябрь 2008 -- Февраль 2011}
        {Удаленный тестировщик}
        {LGPolyREC}
        {Санкт-Петербург}
        {}
        {Из Питерского офиса LG раз в две недели (иногда реже) присылали
        документ с тест-кейсами и приложение для тестирования. Иногда для
        тестирования использовался предоставленный ими телефон.}

\section{Languages}
\cvitemwithcomment{Английский}{Читаю технические тексты}{}
\cvitemwithcomment{Русский}{Родной}{}

\section{Ключевые навыки}
\cvitem
    {Высокий уровень}
    {C}
\cvitem
    {Средний уровень}
    {Архитектура x86, C++, VT-x, gdb}
\cvitem
    {Базовый уровень}
    {Bash, Makefile, asm x86}

\section{Навыки использования ПО}
\cvitem
    {OS}
    {Linux (Arch, Ubuntu), Windows}
\cvitem
    {IDE/Editors}
    {Vim, VisualStudio}
\cvitem
    {VCS}
    {Git}

\section{Публичная активность}
    \cvitem{Habrahabr}{\url{http://habrahabr.ru/users/drnkrq} \newline
        Несколько статей в блоге компании были написаны мной:
        \begin{itemize}
        \item \url{http://habrahabr.ru/company/neobit/blog/173263}
        \item \url{http://habrahabr.ru/company/neobit/blog/211470}
        \item \url{http://habrahabr.ru/company/neobit/blog/203706}
        \item \url{http://habrahabr.ru/company/neobit/blog/197912}
        \end{itemize}
    }
    \cvitem{GitHub}{\url{https://github.com/denisShkirja}}

\end{document}
